
\documentclass[11pt, a4paper]{article}


% --------------------------------------------------------------
% ------------------ Packages ------------------
% --------------------------------------------------------------

\usepackage{academicons} % Add support of academia icons
\usepackage{adjustbox}
\usepackage{amsmath}
\usepackage{amssymb} 
\RequirePackage{array} % Needed to make fixed length table
\usepackage{babel}
\usepackage{bookmark}
\usepackage{booktabs}
\usepackage{calc}
\usepackage{caption}
\usepackage{color} 
\usepackage{colortbl}
\usepackage{currfile-abspath}
\RequirePackage{enumitem} % Needed to handle list environment
\usepackage{eso-pic}
\RequirePackage{etoolbox} % Needed to use a toolbox of programming tools
\usepackage{fancyhdr} % Needed to make header & footer effeciently
\usepackage{float}[h]
\usepackage[T1]{fontenc}
\RequirePackage[quiet]{fontspec} % Needed to manage fonts
\usepackage{geometry}
% \usepackage[margin=1in]{geometry}\usepackage{graphicx}
\usepackage{grffile}
\usepackage{hhline}
\RequirePackage{hyperref} % Needed to deal hyperlink
\RequirePackage{ifxetex} % Needed to use \ifxetex-\else-\fi statement
\usepackage{ifluatex}
\usepackage{import} 
\usepackage{lipsum} 
\usepackage{lmodern}
\usepackage{longtable}
\usepackage{makecell}
\usepackage{makeidx} 
\usepackage{microtype}
\usepackage{morefloats} 
\usepackage{multicol} 
\usepackage{multirow}
\usepackage{overpic}
\RequirePackage{parskip} % Needed to deal a paragraphs
\usepackage{pdflscape}
\RequirePackage{ragged2e} % Needed to handle text alignment
\urlstyle{same} % disable monospaced font for URLs
\RequirePackage{setspace} % Needed to change line spacing in specific environment
\usepackage{siunitx}
\usepackage{subfiles}
\usepackage{tabularx}
\RequirePackage[skins]{tcolorbox} % Needed for the photo ID
\usepackage{tikz}
\usepackage{titlesec}
\usepackage{threeparttable}
\usepackage{threeparttablex}
\usepackage{titling} 
\usepackage[normalem]{ulem}
\RequirePackage{unicode-math} % Needed to manage math fonts
\usepackage{verbatim}
\usepackage{wallpaper}
\usepackage{wrapfig}
\RequirePackage{xifthen} % Needed to use \if-\then-\else statement
\usepackage{xurl} % add URL line breaks if available
\RequirePackage{xcolor} % Needed to manage colors
% \usepackage{xwatermark}

% (https://github.com/posquit0/latex-fontawesome)
\defaultfontfeatures{Extension = .otf}
\RequirePackage{fontawesome}
\RequirePackage[default,opentype]{sourcesanspro}

% Needed to deal hyperlink
% \RequirePackage[hidelinks,unicode]{hyperref}
\hypersetup{
  pdftitle={},
  pdfauthor={},
  pdfsubject={},
  pdfkeywords={},
  colorlinks,
  linkcolor={red!50!black},
  citecolor={blue!50!black},
  urlcolor={blue!80!black}
}

\defaultfontfeatures{Ligatures=TeX} % To support LaTeX quoting style


% --------------------------------------------------------------
% ------------------ Configuration for colors ------------------
% --------------------------------------------------------------

% Gray-scale colors
\definecolor{white}{HTML}{FFFFFF}
\definecolor{black}{HTML}{000000}
\definecolor{darkgray}{HTML}{333333}
\definecolor{gray}{HTML}{5D5D5D}
\definecolor{lightgray}{HTML}{999999}
% Basic colors
\definecolor{green}{HTML}{C2E15F}
\definecolor{orange}{HTML}{FDA333}
\definecolor{purple}{HTML}{D3A4F9}
\definecolor{red}{HTML}{FB4485}
\definecolor{blue}{HTML}{6CE0F1}
% Text colors
\definecolor{darktext}{HTML}{414141}
\colorlet{text}{darkgray}
\colorlet{graytext}{gray}
\colorlet{lighttext}{lightgray}
% Awesome colors
\definecolor{awesome-emerald}{HTML}{00A388}
\definecolor{awesome-skyblue}{HTML}{0395DE}
\definecolor{awesome-red}{HTML}{DC3522}
\definecolor{awesome-pink}{HTML}{EF4089}
\definecolor{awesome-orange}{HTML}{FF6138}
\definecolor{awesome-nephritis}{HTML}{27AE60}
\definecolor{awesome-concrete}{HTML}{95A5A6}
\definecolor{awesome-darknight}{HTML}{131A28}
\colorlet{awesome}{awesome-red}

% Boolean value to switch section color highlighting
\newbool{acvSectionColorHighlight}
\setbool{acvSectionColorHighlight}{true}





% --------------------------------------------------------------
% ------------------ Configuration for layout ------------------
% --------------------------------------------------------------

\geometry{
    a4paper,
    left=20mm,
    right=20mm,
    % headheight=1cm,
    top=1.5cm,
    bottom=2.0cm,
    footskip=0.5cm
  }


%% Header & Footer
% Set offset to each header and footer
\fancyhfoffset{0em}
% Remove head rule
\renewcommand{\headrulewidth}{0pt}
% Clear all header & footer fields
\fancyhf{}


% Enable if you want to make header or footer using fancyhdr
\pagestyle{fancy}


%------------------------------------------------------------------
%------------------------   Set font ------------------------------
%------------------------------------------------------------------

\newcommand{\acvHeaderSocialSep}{\quad\textbar\quad}

\newcommand*{\fontdir}[1][fonts/]{\def\@fontdir{#1}}
\fontdir

\newfontfamily\headerfont[
  Path=\@fontdir,
  UprightFont=*-Regular,
  ItalicFont=*-Italic,
  BoldFont=*-Bold,
  BoldItalicFont=*-BoldItalic,
]{Roboto}

\newfontfamily\headerfontlight[
  Path=\@fontdir,
  UprightFont=*-Thin,
  ItalicFont=*-ThinItalic,
  BoldFont=*-Medium,
  BoldItalicFont=*-MediumItalic,
]{Roboto}



%-------------------------------------------------------------------------------
%                Settings
%-------------------------------------------------------------------------------

% Set false if you don't want to highlight section with awesome color
\setbool{acvSectionColorHighlight}{true}


% Awesome section color
% \def\@sectioncolor#1#2#3{
%   \ifbool{acvSectionColorHighlight}{{\color{awesome}#1#2#3}}{#1#2#3}
% }

% Use to draw horizontal line with specific thickness
% \def\vhrulefill#1{\leavevmode\leaders\hrule\@height#1\hfill \kern\z@}


%  Configure Styles 
\newcommand*{\footerstyle}[1]{{\fontsize{8pt}{1em}\sourcesanspro\scshape\color{lighttext} #1}}
\newcommand*{\sectionstyle}[1]{{\fontsize{16pt}{1em}\sourcesanspro\bfseries\color{text}\@sectioncolor #1}}
\newcommand*{\subsectionstyle}[1]{{\fontsize{12pt}{1em}\sourcesanspro\scshape\textcolor{text}{#1}}}
\newcommand*{\paragraphstyle}{\fontsize{9pt}{1em}\sourcesansprolight\upshape\color{text}}

% Other commands
\newcommand{\comp}{\hspace{0pt}\nolinebreak-\hspace{0pt}} %command for compound words


%--------------------------------------------------------
%	 ---------------- BEGIN DOCUMENT ----------------------
%--------------------------------------------------------


$if(highlighting-macros)$  % highlighting environments, e.g. Shaded
$highlighting-macros$
$endif$

$for(header-includes)$
$header-includes$
$endfor$



\begin{document}

$if(headcolor)$
\definecolor{awesome}{HTML}{$headcolor$}
$else$
\colorlet{awesome}{awesome-red}
$endif$


%	 ---------------- HEADER SECTION ---------------------- 
  \begin{center}

    \fontsize{32pt}{5cm}{
    \vspace{-20em}
    \headerfontlight\color{graytext}{$name$}
    \bfseries\headerfont\color{text}{$surname$}}\\[{.4mm}]
    
    \color{awesome}{\fontsize{7.6pt}{1em}{ \sourcesanspro\scshape $position$}}\\[{.4mm}]
    
    $if(address)$
    \color{lighttext}{\fontsize{8pt}{1em}{ \headerfont\itshape $address$}}\\[{.5mm}]
    $endif$
    
    \fontsize{6.8pt}{1em}\headerfont\color{text}{
    
            $if(phone)$
            \selectfont\faMobile\space $phone$
            $endif$
            $if(email)$
            \acvHeaderSocialSep
            \faEnvelope\space {\color{lighttext}{ $email$}}
            $endif$
            $if(github)$
            \acvHeaderSocialSep
            \faGithub\space \href{https://github.com/$github$}{\color{lighttext}{ $github$}} 
            $endif$
            $if(linkedin)$
            \acvHeaderSocialSep
            \faLinkedinSquare\space \href{https://linkedin.com/in/$linkedin$}{\color{lighttext}{ $linkedin$}} 
            $endif$
            }\\[{5.5mm}]
            
    
    $if(aboutme)$
    \color{darktext}{\fontsize{9pt}{1em}{ \sourcesanspro\itshape $aboutme$}}
    $endif$

\end{center}
  
  % Add header line 
\color{gray}\leavevmode\hrule height 0.9pt \hfill\kern0pt


%	 ---------------- FOOTER ---------------------- 
\vspace{3mm}
\fancyfoot[L]{\fontsize{8pt}{1em}{\sourcesanspro\scshape\color{lighttext} $date$}}
\fancyfoot[C]{\fontsize{8pt}{1em}{\sourcesanspro\scshape\color{lighttext} $footer$}}
\fancyfoot[R]{\fontsize{8pt}{1em}{\sourcesanspro\scshape\color{lighttext} \thepage}}


%-------------------------------------------------------------------------------
%	-------------- Body Section
%------------------------------------------------------------------------------

\renewcommand{\section}[1]{
	\vspace{3mm} % empty space to allocate before paragraph header
	% \sectionstyle{#1}
	\fontsize{16pt}{2em}{\sourcesanspro\scshape\bfseries\color{text}{#1}}  	% \sectionstyle{#1}
	% \phantomsection
	% \color{gray}\leavevmode\leaders\hrule height 0.9pt \hfill\kern0pt % Draw horizontal line with specific thickness
}

% Define a subsection for CV (Level 2 header)
\renewcommand{\subsection}[1]{%
	\vspace{2.5mm}
	\vspace{-3mm}
	\fontsize{12pt}{2em}{\sourcesanspro\scshape\bfseries\textcolor{text}{#1}} 	% \subsectionstyle{#1}
	% \phantomsection
}

% Define a subsection for CV (Level 3 header)
\renewcommand{\subsubsection}[1]{%
	\vspace{2.5mm}
	\vspace{-3mm}
	\fontsize{12pt}{1em}{\sourcesanspro\scshape\textcolor{text}{#1}} % \subsectionstyle{#1}
	% \phantomsection
}

% % Define a paragraph for CV
\newenvironment{cvparagraph}{%
  \vspace{2.5mm}
  \vspace{-3mm}
  \fontsize{9pt}{2em}{\sourcesansprolight\upshape\color{text}}  % \paragraphstyle
}{%
  \par
  \vspace{2mm}
}

% tightlist for - and * markdown bullet points 
\providecommand{\tightlist}{%
  \setlength{\itemsep}{0pt}\setlength{\parskip}{0pt}}

$for(include-before)$
$include-before$
$endfor$

%  \tolerance controls how much white space TeX considers to be "acceptable"
% \emergencystretch configures TeX to use at most 10pt of additional white space per line in order to avoid underfull/overfull lines.
\tolerance=2000 \emergencystretch=10pt

$if(linestretch)$
\setstretch{$linestretch$}
$endif$

$body$

$for(include-after)$
$include-after$
$endfor$




\end{document}
